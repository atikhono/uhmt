\chapter{Introduction}
For years, processor manufacturers have delivered increases in clock rates. While manufacturing technology still improves, physical limitations of semiconductor-based electronics have become a major concern of design. In order for the processors to continue to improve in performance, multi-core design has become necessary.

Unlike the increase in clock frequency, increasing the number of processor cores does not provide automatic benefits for conventional programs. Parallel programming in the conventional style is difficult because computation is mixed with concurrency management, which includes ensuring the correct sequencing of the interactions between different computational executions, and coordinating access to resources that are shared among executions. One of the approaches that addresses the issue of mixed computation and coordination is to separate these concerns. The approach requires explicit parallelisation of an application and specification of component communication in so-called coordination language.

%ak is motivated by KPNs
The concept of a new coordination language \ak\ is described in \cite{astrakahn}. The language defines the coordination behaviour of asynchronous components (boxes) and their orderly interconnection via stream-carrying channels. The \ak\ computation model is based on Kahn process networks, where processes communicate via unbounded FIFO channels. However, in real-world applications, a large number of concurrent processes and their communication facilities share a very limited amount of resources. In order to deal with the issue of application progress, \ak\ attempts to provide a self-regulatory concurrency mechanism based on the concept of communication pressure. Concurrency regulation supposes that several copies of a box may run in parallel, thus \ak\ requires boxes to keep no state. A stateless box knows nothing about synchronisation, therefore data must be synchronised before they are sent to a box. \ak\ provides a synchronisation facility -- a synchroniser -- that is able to store received data and retrieve them for the sole purpose of sending them on, either as they are, or combined with other stored or received data and with trivial data extensions computed by the synchroniser itself.
% ak achieves separation of coordination and synchronisation concerns as well

Boxes are connected to the streaming network with one or two input channels and one or more output channels. Synchronisers may have many input channels. Streaming networks in \ak\ are expressed with four wiring patterns, namely serial and parallel composition, wrap-around connection and serial replication.


    \section{Motivation}
%write about the state of the project. the runtime system needs to be developed before the regulation research.
The \ak\ project is now in its initial state and there is yet no implementation. The playground environment need to be developed for the research towards the self-regulation.

The aim of this thesis is to identify the role of the synchroniser analysis in the project. It includes implementation of the \ak\ synchroniser and integration of the implementation into the runtime system prototype. The output form the infinite replication relies on the synchroniser analysis.

%explain about pressure propagation and statistics
The long-term goal of the \ak\ project is to provide an environment for development of scalable concurrent applications that do not require manual tuning efforts. Though the adaptive self-regulation mechanism is not known to propose something for synchronisers, we try to review the possible strategies for the pressure propagation through synchroniser.


%An Astra\emph{Kahn} approach to adaptive concurrency relies on the concept of communication pressure propagation across a network. Since pressure propagation directly affects proliferation levels of certain vertices, effectiveness of concurrency self-regulation depends largely on correctness of pressure propagation strategy. An important part of this strategy is pressure propagation through synchronisers because, unlike other vertices in Astra\emph{Kahn}, they induce negative pressure that represents a demand for messages in a certain part of a network.

%This project is focused on synchroniser analysis and development of supporting tools. Since the synchroniser is programmed in a dedicated language, a compiler for this language is needed. Target architecture assembly generation is of no concern within the project, thus, the compiler translates a given source code into a C program and then calls available C compiler to generate an executable file. Also, the compiler provides various syntactic and semantic checks.

%In order to propagate pressure through a synchroniser, Astra\emph{Kahn} coordinator needs to know its \emph{transfer function}. The transfer function describes relations between demands of messages on a synchroniser's input and output channels. Derivation of the transfer function is a concern of static analysis, in particular, execution path analysis. Because synchronisers are non-deterministic, there may be potentially an infinite number of possible execution paths and therefore more than one transfer function. It may be impossible to determine an exact execution path at compile time, because it may depend on incoming message content. If synchroniser has an inner loop, the transfer function must consider the number of iterations in this loop. Thus, transfer function derivation relates to induction variable analysis.

%These issues are subject to synchroniser analysis in the project. Once a set of transfer functions is obtained, it is up to TPL how to choose a particular transfer function and propagate pressure through a synchroniser. Also, synchroniser analysis includes passport generation for Constraint Aggregation Layer.


    \section{Contribution}
The contribution of the research is a technique for pressure propagation through synchronisers. This mechanism is an important part of concurrency self-regulating mechanism in Astra\emph{Kahn}. An approach to the problem will be provided in accordance with runtime system regulation policies and a tool for pressure propagation support on runtime will be implemented. The chosen pressure propagation strategy will be tested with a runtime system prototype to decide on its usability for adaptive concurrency regulation. Depending on the result, synchronisation model nondeterminism restrictions and/or strategy changes can be provided to enable efficient concurrency self-regulation on runtime.


    \section{Outline}


%Kahn's fixed-point semantics:
%A network of continous functions is continous
\chapter{Theoretical Background and Related Work}
In this chapter we provide relevant theoretical background in coordination programming and streaming process networks. Particularly, we focus on the approaches to synchronisation in coordination languages. Then, we explain the concepts of \ak\ in more detail and relate it to KPNs and \ak\ predecessor S-Net.


%Approaches to synchronisation.
%Synchronisation facilities in coordination languages.

    \section{Coordination Programming}
The coordination paradigm offers a promising way to address some issues related to the development of efficient parallel systems. Programming a parallel system can be seen as the combination of two activities: the actual computing part comprising a number of processes that manipulate data and a coordination part that is responsible for the communication between the processes.

Basically, coordination is managing dependencies between components. Since the computation is completely separated from the coordination activity, the processes that comprise the former are seen as black boxes. Hence, the actual programming languages used to write the computational code do not play important role in setting up the coordination scheme. Thus, the concept of coordination is closely related to heterogenity.

Existing coordination models\footnote{A coordination model constitutes the entities being coordinated, the means used to ccordinate the entities and the semantic framework the model adheres to} are described in details in the survey \cite{papadopoulos} by G. Papadopoulos and F. Arbab. They argue that these models fall into two major categories of coordination programming, namely either data-driven or control-driven.

The main characteristic of the data-driven coordination models is that coordinated processes are responsible for both examining and manipulating data as well as for coordinating either themselves and/or other processes by invoking the coordination mechanism each language provides. This means that the coordination and computation code are mixed within the process definition. A data-driven coordination language typically offers several coordination primitives which are mixed with the purely computational part of code.

Many data-driven coordination models have evolved around the notion of a shared dataspace. The shared dataspace plays the dual role, being a global data repository and an interprocess communication system. The processes communicate among themselves writing to the shared dataspace and retrieving data from it. Hystorically the first member of this family is Linda \cite{linda}.

Strictly speaking, not all data-driven coordination models follow the above pattern of coordination. Some of them use a message-passing based mechanism (MPI, \cite{mpi}).

Opposite to the data-driven coordination model, the control-driven coordination model achieves almost complete separation of computation and coordination concerns. This is usually achieved by defining a special language that offers facilities for controlling synchronisation, communication, creation and termination of computing components. One of contemporary members of this family is Reo \cite{Reo_Arbab04}.

In Reo the computational components communicate via complex coordinators, or \emph{connectors}. An undirected channel is an atomic connector in Reo. Channels are typed, however, no fixed set of types is assumed. The channel type defines the synchronisational behaviour of the channel with respect to data.

Channels are connected with \emph{nodes}. Nodes have fixed merger-replicator behaviour: the data of one of the incoming channels is propagated to all outgoing channels, without storing or altering the data. If multiple incoming channel can provide data, the node make a nondeterministic choice among them.

Complex connector in Reo is represented as undirected graph contisting of channels and nodes. C. Baier et al. propose \emph{constraint automata} as an operational model for component connectors in Reo \cite{baier_ca}.


    \section{Stream Processing}
The term stream processing refers to the study of a number of disparate systems, such as dataflow systems, reactive systems, signal processing systems, etc. However, conceptually the analysis of each of these systems is usually based on the study of a patricular type of stream processing system. A stream processing system is a system comprised of a collection of processes that compute in parallel and communicate data via channels. The processes are ususally divided into three classes: sources that pass data into the system, filters that perform some atomic computation, and sinks that pass data from the system. Stream processing systems are usually visualised as directed graphs.

An overview of the hystorical development and the discussion of the different techniques for streams programming is presented in the survey by R. Stephens \cite{stephens97}. Stephens identifies that the first type of stream processing systems are dataflow systems. In the first dataflow programming language Lucid \cite{lucid}, each variable is represented as an infinite stream of values. Computation is carried out by defining transformation functions that process these streams. Lucid is possibly the first language to introduce the idea of a filter.

A significant result for concurrency engineering is Kahn's work \cite{kahn74} that outlines the semantics of a simple parallel programming language. Kahn suggests a distributed model of computation where a group of deterministic sequential processes communicate via unbounded FIFO channels under the following assumptions:
\begin{itemize}
\item Channels are the only way for processes to communicate
\item Channels transmit messages within a finite time
\item At any given time a process is either performing computation or waiting for messages on one of its input channels.
\end{itemize}
Kahn proved that the output of the resulting process network is deterministic, i.e. it does not depend on the mutual order of computations at different nodes. The model is now referred to as Kahn Process Network (KPN).

A Kahn process may have multiple input and multiple output channels. Reading from a channel in KPN is blocking, i.e. a process that reads from an empty channel stalls and can only continue when the channel contains sufficient data.  On contrary, writing to a channel is non-blocking and it always succeeds since channels capacity in KPN is unlimited. Processes cannot test an input channel for data availability without consuming the data. KPNs allow arbitrary wiring, i.e. the network may have feedback communication.

In KPNs the number of data elements a process might read from a channel or write to a channel is not restricted. In synchronous dataflow (SDF, \cite{sdf}) the consumption and production rates of a process that is ready to perform the computation are fixed. Hence, an SDF process computes in a synchronised manner with respect to the processes it is connected to.
%Synchronous dataflow is a restriction of KPNs.
%From operational point of view the dataflow model of computation is divided into two basic froms: data-driven (filters compute depending upon the availability of data at their inputs), and demand-driven (filters request data on the input lines when they wish to compute).

A recent SDF implementation in programming languages is StreamIt \cite{streamit}. The basic unit of computation in StreamIt is a user-defined single-input single-output (SISO) block that translates input data sequences to output sequences called a filter. A filter can communicate with neighbouring blocks via FIFO channels. StreamIt imposes structuring on applications with the following structural primitives:
\begin{itemize}
\item \emph{Pipeline} specifies sequential composition of filters,
\item \emph{SplitJoin} specifies parallel composition of filters,
\item and \emph{FeedbackLoop} provides a way to create loop constructs in a streaming network.
\end{itemize}
A StreamIt program is a hierarchical composition of these constructs.

With the restriction to a single input and a single output, a filter is relieved from questions to what extent to synchronise data on multiple input channels and to which of multiple output to send the output.


    \section{S-Net}
S-Net \cite{snet_intro} is a declarative coordination language based on stream processing. S-Net defines the coordination of stateless asynchronous components (boxes) that interact with each other in a streaming network. Boxes are written in conventional languages that are subject to contract with S-Net. Boxes execute fully asynchronously, i.e. a box may consume data as soon as it is available in the input stream. Moreover, boxes are SISO, therefore S-Net achieves a near-complete separation of communication and computation concerns.

Streaming networks are expressed in a hierarchical composition of a fixed set of four combinators, namely serial composition, parallel composition, serial replication and parallel replication. All the combinators except for the serial composition are non-deterministic in a sense that the input data may be reordered in the output stream. For non-deterministic combinators there exist deterministic variants.

S-Net provides a special kind of box, called a synchrocell, that implements barrier synchronisation on streams. Data on streams are organised as records of label-value pairs. A synchrocell maintains an internal state in order to keep a record coming from each of its input streams. Once a synchrocell has collected records from all of its input streams, it merges the records into a single one and emits then the result to the output stream.


%Optionally: synchronisers can be used for monitoring of network
    \section{\ak\ Approach to Streaming Networks}
%In this section we present the concepts of a new coordination language \ak\.
\ak\ is an attempt to combine coordination programming with stream processing in order to provide a component system with concurrency self-regulation. \ak\ defines the coordination behaviour of fully asynchronous components (boxes) and their orderly interconnection via stream-carrying bounded FIFO channels. In \ak\, data on streams are organised as variant records of label-value pairs. Each record constitutes a message.

Like S-Net, \ak\ provides a facility for stream synchronisation in a form of a special component called a synchroniser. The behaviour of a synchroniser is not fixed; instead, it is defined in a dedicated language that is a part of \ak\ paradigm. An \ak\ box is not SISO. Usually it has a single input channel, however, the number of output channels is not restricted. In order to allow the dynamic reconfiguration of \ak\ networks, a box is required to have no state, which it can maintain between two consecutive activations. Since a box is stateless, it does not know to what extent to synchronise the messages it sends to the output channels; this part of work is done by synchronisers. In addition, a synchroniser is not just a tool that is able to combine and store received messages; it can compute some trivial message extensitons. With synchronisers, \ak\ achieves a separation of computation and coordination concerns.

%wiring
\ak\ imposes structuring on streaming networks with a total of four combinators, namely the serial connection, the parallel connection, the wrap-around connection and the serial replication. Network combinators may take either boxes or networks as their operands, hence the network construction is an inductive proccess. 

In the following sections the concepts of \ak\ are explained in more detail.


    \subsubsection{Channels}
A channel carries a segmented stream that consists of message sequences and these sequences may in turn consist of sequences in their own right. In order to mark the beginning and the end of a sequence, \ak\ supports a special kind of message called a segmentation mark.

Segmentation marks can be thought of as brackets. \ak\ requires the static bracketing depth of a stream. Therefore, each message in a given stream is under all circumstances found between the same number of brackets. In this case, further into the sequence brackets can occur only in the following combination:
\[
\underbrace{)\ldots)}_k \underbrace{(\ldots(}_k\,,
\]
where $k \le d$, and $d$ is the number of opening brackets in the beginning of the stream. This combination is denoted as a segmentation mark $\sigma_k$. The bracketing depth $d \ge 0$ characterises the stream-carrying channel\footnote{Indeed, the bracketing depth of a channel that would carry the stream of message lists
\[
(((\;a\underbrace{)(}_{\sigma_1}b\underbrace{))((}_{\sigma_2}c\underbrace{)(}_{\sigma_1}d\;)))
\]
is 3}.


    \subsubsection{Boxes}
Boxes are the atomic building blocks of \ak\ networks that actually perform the computation. An \ak\ box is deterministic in a sense that for every partial input stream it produces a deterministic output stream\footnote{For a function $f(x): \mathcal{I} \to \mathcal{O}$, where $\mathcal{I}$ is the totality of $f(x)$ input streams and $\mathcal{O}$ is the totality of $f(x)$ output streams, $\forall p \in \mathcal{I} \land \forall t:p \cup t \in \mathcal{I} \: : \; f(p \:||\: t) = f(p) \:||\: f(t)$}.

Conceptually, boxes can be specified in any conventional programming language, however, they are subject to a contract that defines acceptable behaviour for boxes. Any guarantees that \ak\ makes\footnote{E.g. regarding ordering of messages in the input channels} are subject to the fulfilment of the contract on behalf of all the boxes. The interface between a box and the \ak\ runtime system is defined by the \ak\ Box-API for each supported box language.

\ak\ declares seven box categories with respect to their algebraic properties and effect of channel segmentation\footnote{The box code does not see the segmentation marks; instead it is declared as having a certain type of bracketing behaviour so that \ak\ can take care of the brackets}:
\begin{description}
\item[Transductor]
A transductor has one input channel and one or more output channels and responds with no more than one output message on each of its output channels. Segmentation marks are passed on to all the output channels of the box, bypassing the box code.

\item[Inductor]
An inductor has one input channel and one or more output channels and responds to a single message from the input channel with a sequence of messages on each of its output channels. Before the input stream is passed to the inductor, each $\sigma_k$ in it with $k > 0$ is replaced by $\sigma_{k+1}$, and a $\sigma_1$ is inserted between every two consecutive data messages. Segmentation marks are bypassed from the input to all the output channels by the coordinator when encountered at the input of the inductor.

\item[Reductor] A reductor implements the reduction operation for a list of input messages. All reductors have a single output channel. \ak\ classifes reductors by the number of input channels and properties of the reduction operation they implement. There exist five classes of reductors:

    \begin{description}
    \item[Dyadic ordered] A dyadic ordered reductor has two input channels. The first input channel is reserved for the initial value. The reduction operator is applied to the messages in the order they arrive on the second input channel

    \item[Dyadic unordered] Same as dyadic ordered except for the reduction operator can be applied to the messages on the second channel in any order without affecting the result

    \item[Monadic ordered and monadic unordered] Same as dyadic reductors except monadic reductors have one input channel. A monadic reductor is only run when two messages are received

    \item[Monadic segmented] A monadic reductor recursively processes an input list of messages that can be segmented into arbitrary sublists until the list is reduced into a single message
    \end{description}
\end{description}


    \subsubsection{Synchronisers}
Synchronisers are non-deterministic finite state machines for joining messages and sending them on to the output channels. \ak\ provides synchronisers a finite memory in order to store received messages.

A synchroniser can have any number of input and output channels. Unlike boxes, synchronisers maintain an internal state and generally accept messages from each input channel in some states, while in any of the other states the channel is blocked until a state transition brings the synchroniser to a state in which messages from the channel are accepted.

A synchroniser can compute trivial extensions for messages. For example, it can add a labeled integer value to a message. It also detects segmantation marks in a an input stream and can change the segmentation of the stream by creating segmentation marks and sending them on to the output channels.

The state transitions of a synchroniser can depend on the content of the current message but never on that of a stored one. In order for the synchroniser to read the field values of the current message, it is matched with a pattern specified within the triggering condition of the transition. In addition, the triggering condition may check the matched values if they are known to be integers. If the message was matched and the integer values are satisfactory, then the transition fires.

The act of sending a message to the output channel is associated with a transition. Once the transition is known to fire, the synchroniser computes the message extensitons, joins the parts of the message together and sends it on to the output channel.

\ak\ provides a dedicated language to describe synchronisers. A synchroniser program is a list of the synchronisers' states. Each state lists the state transitions.



\paragraph{Network Composition}
% TODO 4 or 5 wiring patterns?

Wiring patterns form a small set: only 5 in total, describing connections between
nodes in a hierarchical way. A pattern is applied to its operand networks, either vertices or groups of nodes already wired up with some wiring patterns. The pattern identifies input/output channels of the operand(s) with one another and with the input/output channels of the result. Four out of five patterns are finite, applicable to one or two operands. The fifth one is infinite as it infinitely replicates the single operand network and wires up an infinite chain. That last pattern can also be dynamic, in the sense that the wiring takes place step-by-step at run time, which is the most complex topological and regulatory behaviour available in TPL.

AstraKahn provides a small set of wiring patterns which is sufficient to achieve arbitrary wiring of the vertices.
Static:
- The serial connection
- The parallel connection
- The wrap-around connection
Dynamic:
- Infinite star combinator

every connection that is not serial is a source of non-determinism. non-determinism comes from non-deterministic mergers.

The construction of networks in AstraKahn is hierarchical: vertices are combined into a subnetwork, which in turn can act as a vertex in a larger network, etc.


\paragraph{The type system of \ak\ }

Streams in \ak\ are typed. Data on streams are organised as variant records of label-value pairs. 


the role of the CAL in AstraKahn is similar to the role
of the type system in a conventional (non-coordination) programming language. The CAL is, in a way, a universal type system in the sense that it does not fix the structure and meaning of the type assertions that boxes may choose to import and export. It instead provides a constraint programming framework in which a wide variety of assertions can be formulated. It relies on general-purpose constraint solving as a means of type checking, type inference and most general subtyping.

In data communication the static correctness of the channel demands that the statically guaranteed properties of an output message be sufficient to satisfy the static requirements of its recipients.
a vertex is both an originator of its output messages and the recipient of the input ones, and since the input constraints are the necessary condition for the vertex to operate correctly and hence to guarantee the output properties, the vertex can be abstracted with
respect to its data-transformation behaviour as an implicative statement p ⇒ P. Here p is the conjunction of all the requirements and P is the conjunction of all the guarantees. We will call these implications box passports.

A box is represented as a combination of a source code and a triad of the box name, box category and CAL passport. An AstraKahn compiler performs its first (Constraint Aggregation or CA) pass by only taking the above-mentioned triad and the coordination program written in AstraKahn/TPL. During the CA pass, the topology of the network is extracted from the AstraKahn program, the properties of the synchronisers with respect to input terms (i.e. their “passports” if they had one) are inferred from each synchroniser program, and the process of juxtaposition and constraint solving is performed to instantiate all term variables. As a side effect, a proof is obtained that the constraint system is satisfiable, which indicates that all components have received sufficient assurances to guarantee their output, and consequently the whole program that will be generated next is consistent and type correct.

CAL is based on the Message Definition Language (MDL) which is a language of abstract terms that are built recursively from the ground up. Structurally they are symbolic trees with the following kinds of leaf:
-symbol
-number
-string
-variable
-flag

Terms are built recursively using the following types of constructors:
-tuple
-list
-record
-choice
-switch


Data on streams in \ak\ are organised as choices of records.


%Change the title
    \section{Relation to KPN}
The \ak\ computation model is based on Kahn's model of process networks. Program in Atra\emph{Kahn} is represented as a directed graph of computational processes connected with edges that are stream-carrying channels.

What properties of KPNs do \ak\ inherit? What are the differences?


The semantics of AstraKahn on the TPL is described in terms of structures put in place for the coordinator, i.e. a controlling agent, or indeed a group of agents, responsible for progress and communication of the KPN vertices. The vertices are connected by stream-carrying channels using wiring patterns.


Though \ak\ preserves certain properties of KPNs, it provides the following refinements:
\begin{description}
\item[complete separation of coordination logic from computations]

This refinement leaves computational components stateless and opens opportunities for easier parallelisation. Coordination logic is expressed in special vertices called \emph{synchronisers}. \emph{Synchronisers} are programmed in dedicated programming language;
\item[self-regulatory concurrency mechanism] based on the concept of communication pressure. The mechanism address the issue of application progress under the interpretation when resources are limited. Behavioural classification of computational components is provided along with this mechanism;
\item[separation into independent layers communicating by means of interfaces]

This refinement addresses standard engineering issues such as abstraction, encapsulation and hierarchical development.
\end{description}

As the result of refinement, \ak\ is seen as a construction of three independent layers:
\begin{itemize}
\item Topology and Progress Layer (TPL) defines the topology and provides concurrency regulatory mechanism.
The TPL defines:
1. classes of boxes, their algebraic properties, their effects on channel segmentation and the behaviour under both types of pressure.
2. classes of channels with respect to pressure conductance 3. the language for synchronisers, including:
• the structure of the state, the specification of the state transitions and the associated storage/retrieval behaviour
• pressure creation and conductance • nondeterminism and fairness policies
4. the static wiring patterns 4
5. subnetwork encapsulation facility

\item Constraint Aggregation Layer (CAL) insures type safety all over network with data constraints provided by each component

\item Data Instrumentation Layer (DIL) manages data distribution and concurrent memory access.
\end{itemize}




    \section{Summary}
StreamIt, S-Net and \ak\ share that the networks are constructed with a fixed of combinators. Reo connects the components it coordinates with complex connectors that are constructed of typed channels that express synchronisation.

Which languages separate synchronisation from coordination?
Reo does not.
S-Net (via SISO and synchrocells) and \ak\ (via synchronisers) do. 
StreamIt is SISO and based on the SDF model, where neighbouring processes communicate synchronously.

Unlike S-Net or \ak\, Reo connectors never combine (or process) data items. It just sends them as is.
